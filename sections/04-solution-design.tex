\subsection*{Architektúra systému}

Navrhované riešenie \gls{dms} je koncipované ako viacvrstvová architektúra, ktorá oddeľuje jednotlivé logické časti systému a umožňuje ich nezávislý vývoj, údržbu a budúce rozširovanie. Architektúra je znázornená na Obrázku~\ref{fig:architecture}.

\begin{figure}[h]
    \centering
    \includegraphics[width=0.95\textwidth]{images/architecture.png}
    \caption{Logická architektúra navrhovaného systému správy dokumentov}
    \label{fig:architecture}
\end{figure}

Používateľ pristupuje k systému prostredníctvom webového rozhrania, ktoré je sprístupnené cez webový server zabezpečujúci komunikáciu prostredníctvom protokolu \gls{https}. 

Prezentačná vrstva zabezpečuje používateľské rozhranie systému a umožňuje prácu s dokumentmi, zadávanie údajov a realizáciu schvaľovacích procesov.

Aplikačná vrstva implementuje podnikové pravidlá, riadenie životného cyklu dokumentov, verzovanie a kontrolu prístupových práv na princípe \gls{rbac}. 
Táto vrstva zabezpečuje spracovanie požiadaviek používateľa a komunikáciu s dátovou vrstvou.

Perzistenčná vrstva slúži na uchovávanie štruktúrovaných údajov, ako sú metadáta dokumentov, informácie o projektoch, používateľoch a schvaľovacích procesoch. 
Samotné súbory dokumentov sú ukladané do samostatného úložiska dokumentov, čím sa zabezpečuje oddelenie dátovej logiky od fyzického ukladania súborov.

Asynchrónne spracovanie umožňuje realizáciu časovo náročných operácií, napríklad spracovanie workflow krokov alebo generovanie notifikácií, bez blokovania používateľského rozhrania. 
Komunikácia medzi jednotlivými časťami systému môže byť podporená mechanizmom správ (message queue), čo zvyšuje škálovateľnosť a robustnosť riešenia.

Navrhnutá architektúra poskytuje dostatočnú flexibilitu pre budúce rozšírenie systému a jeho integráciu s ďalšími podnikovými informačnými riešeniami v rámci \gls{pis}.

\clearpage
