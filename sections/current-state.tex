\section{Analýza súčasného stavu}

\subsection*{Charakter spracovania dokumentácie (AS-IS stav)}

Spoločnosť Smer HD, a.s. spracúva ročne približne 4\,000--6\,000 dokumentov rôzneho typu 
(zmluvy, technická dokumentácia, projektové podklady, interné smernice). 
V rámci jedného väčšieho projektu môže vzniknúť 300--800 dokumentov.

V súčasnosti je správa dokumentácie realizovaná kombináciou papierovej archivácie 
a digitálneho úložiska vo forme zdieľaných sieťových priečinkov bez jednotnej klasifikačnej schémy. 
Neexistuje centrálna evidencia dokumentov ani formálne definovaný proces riadenia ich životného cyklu.

Technická dokumentácia (napr. \gls{cad} výkresy) vzniká digitálne, avšak jej evidencia 
je založená primárne na adresárovej štruktúre a manuálnom pomenovávaní súborov. 
Riadené verzovanie nie je implementované – nové verzie dokumentov sú ukladané ako samostatné súbory, 
čo zvyšuje riziko neprehľadnosti a použitia neaktuálnych podkladov.

\subsection*{Procesná analýza toku dokumentov}

Na základe analýzy interných procesov možno súčasný model spracovania dokumentov zhrnúť do nasledovných krokov:

\begin{enumerate}
    \item Vytvorenie dokumentu projektantom alebo administratívnym pracovníkom.
    \item Uloženie dokumentu do lokálneho alebo zdieľaného priečinka.
    \item Oznámenie zmeny prostredníctvom e-mailovej komunikácie.
    \item Manuálne schvaľovanie (odpoveď e-mailom alebo fyzický podpis).
    \item Archivácia dokumentu bez centrálneho registra verzií.
\end{enumerate}

Proces je závislý od individuálnej disciplíny zamestnancov a neobsahuje formálne kontrolné mechanizmy. 
Informácie o stave schvaľovania nie sú systematicky evidované a spätne dohľadateľné.

\subsection*{Kvantifikácia časových strát}

Priemerný čas vyhľadania konkrétneho dokumentu predstavuje 10--15 minút. 
Pri odhadovanom počte približne 500 vyhľadávaní mesačne to znamená:

\[
500 \times 10 \text{ minút} = 5\,000 \text{ minút} \approx 83 \text{ hodín mesačne}
\]

Pri horšom scenári (15 minút):

\[
500 \times 15 \text{ minút} = 7\,500 \text{ minút} \approx 125 \text{ hodín mesačne}
\]

Pri priemernej hodinovej nákladovej cene práce 18 EUR predstavuje mesačný náklad:

\[
83 \times 18 = 1\,494 \text{ EUR}
\]

až

\[
125 \times 18 = 2\,250 \text{ EUR}
\]

Ročne tak možno identifikovať potenciálnu neefektivitu v rozsahu približne 18\,000--27\,000 EUR, 
pričom ide iba o časové straty spojené s vyhľadávaním dokumentov.

\subsection*{Identifikované problémy a ich dopad}

Na základe analýzy boli identifikované nasledovné systémové nedostatky:

\begin{itemize}
    \item \textbf{Absencia centrálneho registra dokumentov} – vedie k duplicitám a strate prehľadu.
    \item \textbf{Nekontrolované verzovanie} – zvyšuje riziko použitia neaktuálnej dokumentácie.
    \item \textbf{Manuálne schvaľovanie} – predlžuje projektové cykly o 1--3 dni.
    \item \textbf{Nedostatočné riadenie prístupových práv} – predstavuje bezpečnostné riziko.
    \item \textbf{Závislosť od konkrétnych pracovníkov} – vytvára úzke miesta (bottleneck efekt).
\end{itemize}

Okrem priamych časových strát existujú aj nepriamo kvantifikovateľné dopady, 
ako je riziko chýb vo výrobe, reputačné škody alebo právne dôsledky v prípade sporových situácií.

\subsection*{Riziková analýza}

Vzhľadom na bezpečnostne citlivý charakter vyrábaných zariadení predstavujú identifikované nedostatky 
významné prevádzkové aj právne riziko.

\begin{table}[ht]
\centering
\begin{tabularx}{\textwidth}{L{6.2cm} C{2.6cm} C{2.4cm} C{3.0cm}}
\toprule
\textbf{Riziko} & \textbf{Pravdepodobnosť} & \textbf{Dopad} & \textbf{Úroveň rizika} \\
\midrule
Použitie neaktuálneho výkresu & Stredná & Vysoký & Kritické \\
Strata dokumentácie & Nízka & Vysoký & Stredné \\
Právny spor bez dôkazov o zmene & Nízka & Vysoký & Stredné \\
Zdržanie projektu kvôli schvaľovaniu & Vysoká & Stredný & Vysoké \\
\bottomrule
\end{tabularx}
\caption{Riziková analýza súčasného stavu}
\end{table}

Najvyššie riziko predstavuje použitie neaktuálnej technickej dokumentácie, 
ktoré môže mať priamy dopad na bezpečnosť a kvalitu realizovaných projektov.

\subsection*{SWOT analýza súčasného stavu}

Pre systematické zhodnotenie situácie bola vypracovaná SWOT analýza:

\begin{table}[ht]
\centering
\begin{tabularx}{\textwidth}{L{0.48\textwidth} L{0.48\textwidth}}
\toprule
\textbf{Silné stránky} & \textbf{Slabé stránky} \\
\midrule
Skúsený technický tím & Neexistencia centrálneho \gls{dms} \\
Projektové know-how & Manuálne schvaľovanie dokumentov \\
Flexibilná organizačná štruktúra & Nekontrolované verzovanie \\
 & Závislosť od jednotlivcov \\
\midrule
\textbf{Príležitosti} & \textbf{Hrozby} \\
\midrule
Digitalizácia procesov & Strata alebo poškodenie dokumentov \\
Zvýšenie efektivity práce & Použitie neaktuálnych výkresov \\
Zlepšenie auditovateľnosti & Právne a bezpečnostné riziká \\
\bottomrule
\end{tabularx}
\caption{SWOT analýza súčasného stavu správy dokumentácie}
\end{table}

Z výsledkov SWOT analýzy vyplýva, že slabé stránky a hrozby priamo súvisia 
s absenciou centrálne riadeného systému správy dokumentov. 
Zároveň však silné stránky podniku, najmä odborná kompetencia tímu a flexibilita organizačnej štruktúry, 
vytvárajú vhodné predpoklady pre úspešnú implementáciu DMS riešenia.

\subsection*{Zhrnutie analýzy súčasného stavu}

Súčasný model správy dokumentácie je funkčný z hľadiska základnej operatívy, 
avšak z pohľadu efektivity, rizík a škálovateľnosti je neudržateľný. 
Identifikované nedostatky predstavujú systematický problém, 
ktorý si vyžaduje koncepčné riešenie v podobe implementácie centralizovaného DMS systému.

\clearpage
