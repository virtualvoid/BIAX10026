\section{Analýza súčasného stavu}

\subsection*{Súčasný spôsob správy dokumentácie}

V súčasnosti spoločnosť Smer HD, a.s. využíva kombináciu papierovej archivácie a základného digitálneho úložiska vo forme zdieľaných sieťových priečinkov. 

Zmluvná dokumentácia je uchovávaná prevažne v tlačenej podobe v archíve administratívneho oddelenia. Technická dokumentácia vzniká digitálne (napr. \gls{cad} výkresy), 
avšak jej evidencia je realizovaná prostredníctvom adresárovej štruktúry bez jednotných pravidiel pomenovávania alebo verzovania.

Schvaľovanie dokumentov prebieha prevažne e-mailovou komunikáciou alebo osobným podpisom vytlačených dokumentov. 
Neexistuje centralizovaný systém evidencie zmien ani automatizované workflow procesy podporené \gls{dms} riešením.

Vyhľadávanie starších dokumentov je časovo náročné a závisí od znalostí konkrétnych zamestnancov. 
V prípade ich neprítomnosti môže dochádzať k výraznému spomaleniu procesov, čo negatívne ovplyvňuje plynulosť realizácie projektov.

\subsection*{Identifikované problémy}

Na základe analýzy aktuálneho stavu možno identifikovať nasledovné problémy:

\begin{itemize}
    \item absencia centrálneho registra dokumentov,
    \item nejednotné pomenovávanie súborov a priečinkov,
    \item neexistujúce systematické verzovanie dokumentov,
    \item manuálne a neformálne schvaľovacie procesy,
    \item obmedzená kontrola prístupových práv,
    \item vysoká závislosť od konkrétnych pracovníkov.
\end{itemize}

Tieto nedostatky vedú k zníženej efektivite práce, zvyšujú pravdepodobnosť chýb pri práci s technickou dokumentáciou a komplikujú dohľadateľnosť historických údajov.

\subsection*{Riziká vyplývajúce zo súčasného stavu}

Nedostatočne riadená správa dokumentácie môže viesť k viacerým rizikám:

\begin{itemize}
    \item použitie neaktuálnej verzie technickej dokumentácie vo výrobe,
    \item strata alebo poškodenie papierových archívov,
    \item nemožnosť spätnej dohľadateľnosti zmien,
    \item právne riziká pri sporoch so zákazníkmi,
    \item problémy pri auditoch alebo bezpečnostných kontrolách.
\end{itemize}

Vzhľadom na bezpečnostne citlivý charakter vyrábaných zariadení predstavujú uvedené riziká významnú hrozbu pre reputáciu aj ekonomickú stabilitu podniku.

\subsection*{SWOT analýza súčasného stavu}

Pre systematické zhodnotenie situácie bola vypracovaná SWOT analýza:

\begin{table}[h]
\centering
\begin{tabular}{p{6cm} p{6cm}}
\toprule
\textbf{Silné stránky} & \textbf{Slabé stránky} \\
\midrule
Skúsený technický tím &
Neexistencia centrálneho \gls{dms} \\
Projektové know-how &
Manuálne schvaľovanie dokumentov \\
Flexibilná organizačná štruktúra &
Nekontrolované verzovanie \\
 & Závislosť od jednotlivcov \\
\midrule
\textbf{Príležitosti} & \textbf{Hrozby} \\
\midrule
Digitalizácia procesov &
Strata alebo poškodenie dokumentov \\
Zvýšenie efektivity práce &
Použitie neaktuálnych výkresov \\
Zlepšenie auditovateľnosti &
Právne a bezpečnostné riziká \\
\bottomrule
\end{tabular}
\caption{SWOT analýza súčasného stavu správy dokumentácie}
\end{table}

Analýza poukazuje na skutočnosť, že hoci podnik disponuje silným odborným zázemím, absentuje systematická informačná podpora správy dokumentov. 
Digitalizácia a implementácia vhodného \gls{dms} systému predstavuje prirodzený krok smerujúci k zvýšeniu efektivity, transparentnosti a zníženiu prevádzkových rizík.

\clearpage