\section{Implementačný plán}

Implementácia \gls{dms} v spoločnosti Smer HD, a.s. musí prebiehať riadeným spôsobom tak, aby nedošlo k narušeniu prebiehajúcich projektov a výrobných procesov. 
Navrhovaná implementácia je rozdelená do niekoľkých logických fáz, ktoré umožnia postupný prechod zo súčasného stavu na cieľové riešenie.

\subsection*{Fáza 1 – Analýza a detailná špecifikácia}

V úvodnej fáze je potrebné podrobne špecifikovať požiadavky jednotlivých oddelení a definovať presné pravidlá práce s dokumentáciou. Súčasťou tejto fázy je:

\begin{itemize}
    \item definovanie kategórií dokumentov,
    \item návrh štruktúry metadát,
    \item identifikácia schvaľovacích procesov,
    \item návrh používateľských rolí a prístupových práv na princípe \gls{rbac}.
\end{itemize}

Táto fáza predstavuje základ pre úspešnú implementáciu a minimalizuje riziko neskorších úprav systému.

\subsection*{Fáza 2 – Návrh a konfigurácia systému}

Na základe analýzy sa pripraví konkrétny návrh konfigurácie systému. V tejto fáze sa realizujú najmä:

\begin{itemize}
    \item návrh dátového modelu systému,
    \item implementácia základných workflow procesov,
    \item definovanie štruktúry projektov a registratúrnych celkov,
    \item prispôsobenie používateľského rozhrania potrebám firmy.
\end{itemize}

Cieľom tejto fázy je vytvoriť funkčný základ riešenia, ktorý bude reflektovať organizačné a procesné špecifiká podniku.

\subsection*{Fáza 3 – Migrácia existujúcej dokumentácie}

Jedným z kľúčových krokov je postupná migrácia existujúcej papierovej a digitálnej dokumentácie do nového systému. Tento proces zahŕňa:

\begin{itemize}
    \item digitalizáciu papierových dokumentov,
    \item triedenie a kategorizáciu súborov,
    \item kontrolu úplnosti a správnosti údajov,
    \item testovanie vyhľadávania a prístupových práv.
\end{itemize}

Migrácia bude realizovaná postupne podľa jednotlivých projektov, aby sa minimalizovalo prevádzkové riziko a zabezpečila kontinuita činností podniku.

\subsection*{Fáza 4 – Testovanie a pilotná prevádzka}

Pred nasadením do ostrej prevádzky je potrebné systém otestovať v pilotnom režime. Pilotná prevádzka môže byť realizovaná na vybranom projekte alebo oddelení.

Cieľom tejto fázy je:

\begin{itemize}
    \item overenie funkčnosti workflow procesov,
    \item identifikácia používateľských problémov,
    \item úprava nastavení systému na základe spätnej väzby.
\end{itemize}

Pilotná prevádzka umožní odhaliť potenciálne nedostatky ešte pred plošným nasadením riešenia.

\subsection*{Fáza 5 – Ostrá prevádzka a školenie zamestnancov}

Po úspešnom ukončení pilotnej fázy sa systém nasadí do plnej prevádzky. Súčasťou implementácie je školenie zamestnancov, ktoré zabezpečí správne používanie systému a zníži odpor voči zmene pracovných postupov.

V tejto fáze sa zároveň nastaví proces pravidelnej podpory, údržby a aktualizácie systému.

\subsection*{Časový rámec implementácie}

Predpokladaná dĺžka implementácie je približne 4 až 6 mesiacov v závislosti od rozsahu migrácie dokumentácie, komplexnosti procesov a pripravenosti zamestnancov na zmenu pracovných postupov.

\subsection*{Riziká implementácie}

Implementácia systému môže byť sprevádzaná nasledovnými rizikami:

\begin{itemize}
    \item odpor zamestnancov voči zmene pracovných postupov,
    \item podcenenie náročnosti migrácie dokumentov,
    \item nedostatočné školenie používateľov,
    \item dočasné spomalenie práce počas prechodného obdobia.
\end{itemize}

Tieto riziká možno minimalizovať postupným nasadzovaním systému, jasnou komunikáciou zmien, zapojením kľúčových pracovníkov do prípravnej fázy a dôsledným zaškolením používateľov.

\clearpage