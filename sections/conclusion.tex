\section{Záver}

Cieľom tejto práce bolo analyzovať potreby modelovej spoločnosti Smer HD, a.s. a navrhnúť implementáciu \gls{dms} ako súčasti \gls{pis}. 
Na základe analýzy súčasného stavu bolo identifikované, že podnik síce disponuje odborným a technickým potenciálom, avšak správa dokumentácie nie je systematicky podporená vhodným informačným riešením.

Projektovo orientovaný charakter výroby horských dráh generuje rozsiahle množstvo technickej, zmluvnej a administratívnej dokumentácie. 
Absencia centralizovaného systému, nejednotné verzovanie a manuálne schvaľovacie procesy predstavujú významné riziká z hľadiska efektivity, bezpečnosti aj právnej istoty.

Navrhnuté riešenie založené na viacvrstvovej architektúre poskytuje podniku nástroj na centralizovanú evidenciu dokumentov, riadenie prístupových práv, 
podporu workflow procesov a zabezpečenie auditovateľnosti zmien. Implementačný plán bol rozdelený do logických fáz s dôrazom na postupnú migráciu dokumentácie a minimalizáciu prevádzkových rizík.

Z pohľadu dlhodobého rozvoja možno implementáciu \gls{dms} vnímať ako prvý krok k širšej digitalizácii podniku. 
Systematické riadenie dokumentácie vytvára predpoklady pre efektívnejšie riadenie projektov, lepšiu využiteľnosť interného know-how a budúcu integráciu s ďalšími podnikovými systémami v rámci \gls{pis}.

Možno konštatovať, že navrhovaná implementácia predstavuje realistické a efektívne riešenie, ktoré reaguje na identifikované problémy a podporuje strategický rozvoj spoločnosti.

\clearpage
