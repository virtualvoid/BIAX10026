\section*{Záver}
\addcontentsline{toc}{section}{Záver}

Cieľom tejto práce bolo analyzovať potreby modelovej spoločnosti Smer HD, a.s. 
v oblasti správy dokumentácie a navrhnúť implementáciu systému správy dokumentov 
ako súčasti podnikového informačného systému. 

V teoretickej časti boli predstavené základné východiská podnikového informačného systému, 
jeho postavenie v organizácii a význam špecializovaných riešení typu \gls{dms}. 
Pozornosť bola venovaná životnému cyklu informačného systému, 
architektonickým prístupom k jeho implementácii a významu digitalizácie 
v kontexte zvyšovania efektivity a konkurencieschopnosti podniku.

Analýza súčasného stavu spoločnosti Smer HD, a.s. poukázala na viaceré nedostatky 
v oblasti evidencie, verzovania a schvaľovania dokumentácie. 
Identifikované problémy – decentralizované ukladanie dokumentov, 
manuálne schvaľovanie, absencia auditnej stopy a riziko použitia neaktuálnych verzií – 
predstavujú nielen prevádzkové, ale aj právne a ekonomické riziká.

Na základe tejto analýzy boli definované funkčné, nefunkčné, bezpečnostné a integračné požiadavky 
a následne navrhnutá viacvrstvová architektúra riešenia. 
Návrh systému reflektuje potrebu centrálnej evidencie dokumentov, 
riadeného verzovania, workflow mechanizmov a riadenia prístupových práv na princípe \gls{rbac}. 
Architektúra zároveň vytvára predpoklady pre budúcu integráciu s ďalšími podnikovými systémami.

Súčasťou práce bol aj návrh implementačného plánu rozdeleného do piatich fáz, 
vrátane analýzy, konfigurácie systému, migrácie dokumentácie, pilotnej prevádzky a stabilizácie. 
Takto štruktúrovaný prístup minimalizuje prevádzkové riziko 
a umožňuje postupný prechod zo súčasného stavu do cieľového riešenia.

Analýza prínosov ukázala, že implementácia \gls{dms} môže priniesť významné operatívne, 
ekonomické aj strategické benefity. 
Kvantifikovaný model úspory pracovného času naznačuje, 
že investícia do systému je v strednodobom horizonte ekonomicky opodstatnená. 
Zároveň boli identifikované projektové, technické, organizačné a bezpečnostné riziká, 
ktoré je možné primerane riadiť pomocou vhodných opatrení.

Na základe vykonanej analýzy možno konštatovať, že stanovený cieľ práce bol splnený. 
Bola identifikovaná problematika správy dokumentácie v podniku, 
navrhnuté systematické riešenie vrátane architektúry, dátového modelu a implementačného plánu 
a posúdené jeho ekonomické a organizačné dopady.

Implementácia systému správy dokumentov predstavuje pre spoločnosť Smer HD, a.s. 
strategický krok smerom k systematickej digitalizácii procesov, 
zvýšeniu transparentnosti a zníženiu prevádzkových rizík. 
Z dlhodobého hľadiska môže byť navrhnuté riešenie základom 
pre rozšírenie podnikového informačného systému o ďalšie moduly, 
napríklad integráciu s \gls{erp} systémom alebo rozšírenie o registratúrne funkcie.

Budúci rozvoj systému môže zahŕňať implementáciu pokročilého fulltextového vyhľadávania, 
rozšírenie integračných rozhraní, 
využitie cloudovej infraštruktúry alebo zavedenie pokročilých analytických nástrojov 
na vyhodnocovanie procesnej efektivity.

Navrhované riešenie tak nepredstavuje izolovaný technologický projekt, 
ale koncepčný základ pre modernizáciu informačného prostredia podniku 
a posilnenie jeho dlhodobej konkurencieschopnosti.

\clearpage