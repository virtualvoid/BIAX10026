\section{Identifikácia požiadaviek na informačný systém}

Na základe analýzy súčasného stavu boli identifikované požiadavky na implementáciu \gls{dms}, 
ktorý bude reflektovať špecifiká projektovo orientovaného výrobného podniku. 
Požiadavky vychádzajú z identifikovaných procesných nedostatkov, rizík a ekonomických dopadov 
popísaných v predchádzajúcej kapitole.

Požiadavky boli rozdelené do štyroch kategórií: funkčné, nefunkčné, bezpečnostné a integračné. 
Pre zvýšenie prehľadnosti a určenie implementačnej priority bola použitá metóda MoSCoW 
(Must, Should, Could).

\subsection*{Prehľad a priorizácia požiadaviek (MoSCoW)}

\begin{table}[ht]
\centering
\small
\begin{tabularx}{\textwidth}{C{1.2cm} L{9.0cm} C{2.4cm} C{2.2cm}}
\toprule
\textbf{ID} & \textbf{Požiadavka} & \textbf{Kategória} & \textbf{Priorita} \\
\midrule
F1 & Centrálna evidencia dokumentov a väzba na projekt (jednoznačná identifikácia, typ dokumentu, metadáta) & Funkčná & Must \\
F2 & Riadené verzovanie dokumentov vrátane histórie zmien a označenia platnej verzie & Funkčná & Must \\
F3 & Workflow schvaľovanie (kroky, úlohy, rozhodnutia, stav dokumentu) \cite{dumas} & Funkčná & Must \\
F4 & Vyhľadávanie podľa viacerých kritérií (projekt, typ, dátum, zodpovedná osoba) & Funkčná & Should \\
F5 & Podpora šablón a štandardizovaných typov dokumentov & Funkčná & Could \\
\midrule
N1 & Dostupnosť v internom sieťovom prostredí a používateľsky zrozumiteľné rozhranie & Nefunkčná & Must \\
N2 & Spoľahlivosť, pravidelné zálohovanie a podpora obnovy dát & Nefunkčná & Must \\
N3 & Výkon pri práci s rozsiahlymi technickými súbormi & Nefunkčná & Should \\
\midrule
B1 & Riadenie prístupových práv na princípe \gls{rbac} & Bezpečnostná & Must \\
B2 & Auditná stopa operácií (kto, kedy, čo zmenil) & Bezpečnostná & Must \\
B3 & Ochrana proti neoprávnenému zásahu a úniku údajov & Bezpečnostná & Must \\
\midrule
I1 & Možnosť budúcej integrácie s \gls{erp}/ekonomickým systémom & Integračná & Could \\
I2 & API alebo exportné mechanizmy pre integráciu v rámci \gls{pis} & Integračná & Could \\
\bottomrule
\end{tabularx}
\caption{Prehľad požiadaviek a ich priorizácia metódou MoSCoW}
\end{table}

\subsection*{Funkčné požiadavky}

Základnou požiadavkou je zavedenie centrálnej evidencie dokumentov (F1), 
ktorá odstráni problém decentralizovaného ukladania a nejednotnej klasifikácie. 
Každý dokument musí byť jednoznačne identifikovateľný a priradený ku konkrétnemu projektu.

Kľúčovým prvkom je riadené verzovanie (F2), ktoré zabezpečí uchovávanie histórie zmien 
a jednoznačné označenie aktuálnej verzie dokumentu. 
Táto požiadavka priamo reaguje na identifikované riziko použitia neaktuálnej dokumentácie.

Workflow mechanizmus (F3) umožní formalizovať schvaľovací proces, 
znížiť závislosť od e-mailovej komunikácie a zabezpečiť preukázateľnú evidenciu rozhodnutí.

Vyhľadávanie podľa viacerých kritérií (F4) má minimalizovať časové straty identifikované v analýze AS-IS stavu.

\subsection*{Nefunkčné požiadavky}

Systém musí byť dostupný všetkým oprávneným používateľom v rámci interného prostredia (N1) 
a poskytovať intuitívne používateľské rozhranie.

Spoľahlivosť a ochrana dát (N2) predstavujú kritický faktor vzhľadom na bezpečnostnú povahu dokumentácie. 
Systém musí podporovať pravidelné zálohovanie a obnovu dát.

Požiadavka na výkon (N3) reflektuje prácu s rozsiahlymi technickými súbormi 
a potrebu zachovania plynulosti práce používateľov.

\subsection*{Bezpečnostné požiadavky}

Riadenie prístupových práv na princípe \gls{rbac} (B1) umožní definovať roly 
a obmedziť prístup podľa projektových tímov a organizačnej štruktúry.

Auditná stopa (B2) zabezpečí spätnú dohľadateľnosť zmien, 
čo je významné z hľadiska interných kontrol aj potenciálnych sporových situácií.

Ochrana pred neoprávneným zásahom (B3) zahŕňa kontrolu prístupov, 
ochranu dát a minimalizáciu rizika úniku citlivých informácií.

\subsection*{Integračné požiadavky}

Aj keď primárnym cieľom je riešenie správy dokumentov, 
systém musí umožniť budúcu integráciu s ekonomickým alebo \gls{erp} systémom (I1). 
Prepojenie dokumentov s projektmi, zákazkami alebo dokladmi 
zvýši transparentnosť riadenia podniku.

Dostupnosť API alebo exportných mechanizmov (I2) umožní postupnú integráciu 
v rámci širšieho podnikového informačného systému.

\subsection*{Zdôvodnenie voľby DMS riešenia}

Na základe analýzy alternatív (ERP, ECM, samostatné DMS) možno konštatovať, 
že primárnym problémom podniku je systematická správa dokumentácie, 
nie komplexné riadenie výrobných alebo finančných procesov.

Implementácia samostatného \gls{dms} predstavuje realistický a postupne implementovateľný krok, 
ktorý rieši najkritickejšie identifikované problémy 
a zároveň vytvára základ pre budúcu integráciu s ďalšími systémami.

\subsection*{Ekonomické aspekty implementácie}

Implementácia \gls{dms} predstavuje investičné rozhodnutie s priamym dopadom na efektivitu práce. 
Na základe analýzy časových strát možno očakávať významnú redukciu neproduktívneho času 
spojeného s vyhľadávaním dokumentov a manuálnym schvaľovaním.

Investičné náklady zahŕňajú:

\begin{itemize}
    \item softvérové licencie alebo vývoj riešenia,
    \item infraštruktúru a zálohovanie,
    \item migráciu existujúcej dokumentácie,
    \item školenie používateľov.
\end{itemize}

Očakávané prínosy zahŕňajú zvýšenie efektivity, zníženie rizík 
a podporu transparentného riadenia projektov.

\subsection*{Zhrnutie požiadaviek}

Identifikované požiadavky predstavujú formálny základ pre návrh architektúry riešenia. 
Priorita kategórie Must definuje minimálny rozsah funkcionality, 
ktorý musí byť implementovaný pre odstránenie kľúčových nedostatkov súčasného stavu.

\clearpage
