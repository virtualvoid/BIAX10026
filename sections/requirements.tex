\section{Identifikácia požiadaviek na informačný systém}

Na základe analýzy súčasného stavu boli identifikované požiadavky na implementáciu \gls{dms}, ktorý bude reflektovať špecifiká projektovo orientovaného výrobného podniku.

Požiadavky možno rozdeliť do viacerých kategórií: funkčné, nefunkčné, bezpečnostné a integračné.

\subsection*{Funkčné požiadavky}

Funkčné požiadavky vyjadrujú, aké činnosti musí systém podporovať.

Systém musí umožňovať centrálne evidovanie všetkých dokumentov vznikajúcich v rámci projektu. Každý dokument musí byť jednoznačne identifikovateľný, 
zaraditeľný k príslušnému projektu a kategorizovaný podľa typu (technická dokumentácia, zmluva, certifikát a pod.).

Súčasťou systému musí byť mechanizmus verzovania dokumentov, ktorý umožní sledovanie zmien a uchovávanie histórie úprav. Používateľ musí mať možnosť identifikovať aktuálnu platnú verziu dokumentu.

Dôležitou funkciou je podpora schvaľovacích procesov prostredníctvom workflow mechanizmov \cite{dumas}. 
Systém musí umožňovať definovanie schvaľovacích krokov, priraďovanie úloh konkrétnym používateľom a evidenciu rozhodnutí.

Systém musí podporovať vyhľadávanie dokumentov podľa viacerých kritérií, napríklad podľa projektu, typu dokumentu, dátumu vytvorenia alebo zodpovednej osoby.

\subsection*{Nefunkčné požiadavky}

Nefunkčné požiadavky definujú kvalitatívne vlastnosti systému.

Systém musí byť dostupný pre všetkých oprávnených používateľov prostredníctvom interného sieťového prostredia podniku. 
Používateľské rozhranie musí byť intuitívne a zrozumiteľné aj pre technicky menej zdatných pracovníkov.

Dôležitou požiadavkou je spoľahlivosť systému a zabezpečenie ochrany dát proti strate. Systém musí podporovať pravidelné zálohovanie a obnovu dát.

Z pohľadu výkonu musí byť systém schopný pracovať s rozsiahlymi technickými súbormi bez výrazného spomalenia práce používateľov.

\subsection*{Bezpečnostné požiadavky}

Vzhľadom na citlivosť technickej a zmluvnej dokumentácie je nevyhnutné zabezpečiť riadenie prístupových práv na princípe \gls{rbac}. 
Systém musí umožňovať definovanie používateľských rolí a obmedzenie prístupu k dokumentom podľa organizačnej štruktúry.

Každá zmena dokumentu musí byť zaznamenaná v auditnej stope, ktorá umožní spätne identifikovať autora zmeny, čas úpravy a charakter vykonanej operácie.

Systém musí zabezpečiť ochranu pred neoprávneným zásahom do dokumentácie a minimalizovať riziko úniku citlivých údajov.

\subsection*{Integračné požiadavky}

Hoci hlavnou oblasťou riešenia je správa dokumentov, systém by mal umožňovať budúcu integráciu s ďalšími podnikovými systémami, najmä s ekonomickým softvérom alebo \gls{erp} riešením.

Integrácia by mala umožniť prepojenie dokumentov s konkrétnymi projektmi, zákazkami alebo fakturačnými dokladmi, čím sa zvýši prehľadnosť a efektivita riadenia podniku. 
Takáto integrácia podporí komplexné fungovanie \gls{pis} v rámci organizácie.

\subsection*{Porovnanie možných prístupov riešenia}

Pri rozhodovaní o implementácii informačného riešenia je potrebné zvážiť viaceré alternatívy. 
Pre spoločnosť Smer HD, a.s. prichádzajú do úvahy najmä nasledovné prístupy:

\textbf{1. Implementácia komplexného ERP riešenia}

ERP systémy poskytujú integrované riadenie financií, výroby, skladového hospodárstva a ďalších procesov. 
Rozšírené ERP riešenia často obsahujú aj modul správy dokumentov. 

Nevýhodou takéhoto prístupu môže byť vysoká finančná náročnosť, dlhšia implementácia a implementácia funkcionalít, ktoré podnik v súčasnosti bezprostredne nepotrebuje.

\textbf{2. Implementácia ECM systému}

ECM riešenia poskytujú širší rámec správy obsahu vrátane digitálnych archívov, správy záznamov a často aj integrácie s inými podnikovými systémami. 
Pre menší projektovo orientovaný podnik však môže byť implementácia komplexného ECM riešenia organizačne aj finančne nadmerná.

\textbf{3. Implementácia samostatného DMS riešenia}

Samostatný \gls{dms} predstavuje cielené riešenie zamerané na evidenciu, verzovanie a riadenie životného cyklu dokumentov. 
Vzhľadom na identifikované problémy v oblasti správy dokumentácie predstavuje tento prístup primeraný kompromis medzi funkčnosťou, nákladmi a implementačnou náročnosťou.

\subsection*{Zdôvodnenie výberu riešenia}

Na základe analýzy potrieb podniku možno konštatovať, že primárnym problémom spoločnosti nie je komplexné riadenie výroby alebo financií, ale systematická správa dokumentácie.

Z tohto dôvodu je implementácia \gls{dms} vhodnejším a realistickejším prvým krokom v rámci postupnej digitalizácie podniku. 
Takýto prístup umožňuje riešiť najkritickejšiu oblasť s primeranými nákladmi a zároveň vytvára základ pre budúcu integráciu s ERP alebo inými podnikovými systémami.

\subsection*{Porovnanie možných prístupov riešenia}

Pri rozhodovaní o implementácii informačného riešenia je potrebné zvážiť viaceré alternatívy. 
Pre spoločnosť Smer HD, a.s. prichádzajú do úvahy najmä nasledovné prístupy:

\textbf{1. Implementácia komplexného ERP riešenia}

ERP systémy poskytujú integrované riadenie financií, výroby, skladového hospodárstva a ďalších procesov. 
Rozšírené ERP riešenia často obsahujú aj modul správy dokumentov. 

Nevýhodou takéhoto prístupu môže byť vysoká finančná náročnosť, dlhšia implementácia a implementácia funkcionalít, ktoré podnik v súčasnosti bezprostredne nepotrebuje.

\textbf{2. Implementácia ECM systému}

ECM riešenia poskytujú širší rámec správy obsahu vrátane digitálnych archívov, správy záznamov a často aj integrácie s inými podnikovými systémami. 
Pre menší projektovo orientovaný podnik však môže byť implementácia komplexného ECM riešenia organizačne aj finančne nadmerná.

\textbf{3. Implementácia samostatného DMS riešenia}

Samostatný \gls{dms} predstavuje cielené riešenie zamerané na evidenciu, verzovanie a riadenie životného cyklu dokumentov. 
Vzhľadom na identifikované problémy v oblasti správy dokumentácie predstavuje tento prístup primeraný kompromis medzi funkčnosťou, nákladmi a implementačnou náročnosťou.

\subsection*{Zdôvodnenie výberu riešenia}

Na základe analýzy potrieb podniku možno konštatovať, že primárnym problémom spoločnosti nie je komplexné riadenie výroby alebo financií, ale systematická správa dokumentácie.

Z tohto dôvodu je implementácia \gls{dms} vhodnejším a realistickejším prvým krokom v rámci postupnej digitalizácie podniku. 
Takýto prístup umožňuje riešiť najkritickejšiu oblasť s primeranými nákladmi a zároveň vytvára základ pre budúcu integráciu s ERP alebo inými podnikovými systémami.

\clearpage