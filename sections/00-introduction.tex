\section{Úvod}

Digitalizácia podnikových procesov predstavuje jeden z kľúčových faktorov konkurencieschopnosti moderných organizácií. 
Rastúca komplexnosť riadenia, tlak na efektivitu, transparentnosť a kontrolu nákladov vedú podniky k implementácii podporných informačných systémov, 
ktoré umožňujú efektívne spracovanie a riadenie údajov.

Podnikový informačný systém (PIS) možno charakterizovať ako integrovaný súbor softvérových nástrojov, databáz a technologických komponentov, 
ktoré podporujú plánovanie, evidenciu, riadenie a kontrolu podnikových procesov. Úlohou takéhoto systému je zabezpečiť dostupnosť presných 
a aktuálnych informácií pre manažment aj operatívnych pracovníkov, čím sa znižuje riziko chýb, zvyšuje sa produktivita a zlepšuje sa rozhodovací proces.

\subsection*{Klasifikácia podnikových informačných systémov}

Podnikové informačné systémy možno rozdeliť podľa oblasti, ktorú podporujú. Medzi najvýznamnejšie kategórie patria:

\begin{itemize}
    \item \textbf{ERP (Enterprise Resource Planning)} – systémy určené na plánovanie a riadenie podnikových zdrojov, najmä v oblasti financií, výroby, skladového hospodárstva a personalistiky.
    \item \textbf{CRM (Customer Relationship Management)} – systémy zamerané na riadenie vzťahov so zákazníkmi, evidenciu obchodných prípadov a podporu marketingových aktivít.
    \item \textbf{SCM (Supply Chain Management)} – systémy podporujúce riadenie dodávateľského reťazca a optimalizáciu logistických procesov.
    \item \textbf{ECM/DMS (Enterprise Content Management / Document Management System)} – systémy určené na správu dokumentov a digitálneho obsahu v rámci organizácie.
\end{itemize}

V praxi sú tieto systémy často vzájomne prepojené a tvoria integrované prostredie, ktoré zabezpečuje komplexnú podporu podnikových činností.

\subsection*{Význam systémov správy dokumentov}

Systémy správy dokumentov (\gls{dms}) predstavujú špecializovanú kategóriu podnikových informačných systémov, ktorých cieľom je evidencia, archivácia, verzovanie a riadenie životného cyklu dokumentov. 
V organizáciách, ktoré pracujú s rozsiahlym množstvom technickej, projektovej alebo zmluvnej dokumentácie, zohrávajú zásadnú úlohu.

Moderný DMS systém umožňuje:

\begin{itemize}
    \item centrálne a bezpečné uloženie dokumentov,
    \item riadenie prístupových práv podľa organizačnej štruktúry,
    \item verzovanie dokumentov a sledovanie zmien,
    \item automatizáciu schvaľovacích procesov prostredníctvom workflow mechanizmov,
    \item vytváranie auditnej stopy pre potreby kontroly a dohľadu.
\end{itemize}

Implementácia DMS systému prispieva k zníženiu administratívnej záťaže, minimalizácii rizika straty dokumentov a zvýšeniu efektivity práce zamestnancov. 
Zároveň umožňuje lepšiu kontrolu nad dodržiavaním interných smerníc a legislatívnych požiadaviek.

\subsection*{Cieľ práce}

Cieľom tejto práce je analyzovať potreby modelovej spoločnosti \textbf{Smer HD, a.s.}, ktorá sa zaoberá výrobou horských dráh, 
a navrhnúť implementáciu systému správy dokumentov ako súčasti podnikového informačného systému. 

Práca sa zameriava na analýzu súčasného stavu správy dokumentácie v podniku, identifikáciu problémov vyplývajúcich z papierovej 
a čiastočne neorganizovanej digitálnej evidencie a návrh riešenia vrátane implementačného plánu a očakávaných prínosov.

\clearpage
