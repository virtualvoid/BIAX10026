\section{Očakávané prínosy implementácie}

Implementácia \gls{dms} v spoločnosti Smer HD, a.s. nepredstavuje iba technologickú zmenu, ale zásah do spôsobu riadenia informácií v podniku. 
Zavedenie centralizovaného systému správy dokumentov môže mať významný dopad na efektivitu, bezpečnosť a transparentnosť procesov.

\subsection*{Zvýšenie efektivity práce}

Centralizovaná evidencia dokumentov a možnosť ich rýchleho vyhľadávania povedie k zníženiu času potrebného na administratívne úkony. 
Zamestnanci nebudú odkázaní na manuálne prehľadávanie priečinkov alebo papierových archívov, čím sa zníži časová náročnosť práce s dokumentáciou.

Automatizácia schvaľovacích procesov prostredníctvom workflow mechanizmov \cite{dumas} zároveň obmedzí potrebu neformálnej e-mailovej komunikácie a zvýši prehľadnosť rozhodovacích procesov.

\subsection*{Zníženie prevádzkových a právnych rizík}

Presné verzovanie dokumentov a existencia auditnej stopy výrazne znižujú riziko použitia neaktuálnych technických podkladov vo výrobe. 
V prípade kontroly alebo sporu so zákazníkom bude podnik schopný jednoznačne preukázať históriu zmien dokumentácie.

Digitalizácia archívu zároveň minimalizuje riziko straty alebo poškodenia papierových dokumentov a podporuje dlhodobú archiváciu údajov.

\subsection*{Zlepšenie riadenia projektov}

Prepojenie dokumentácie s konkrétnymi projektmi umožní lepšiu kontrolu nad priebehom realizácie zákaziek. Projektoví manažéri získajú okamžitý prehľad o stave schvaľovania dokumentov a dostupnosti technických podkladov.

Transparentnejší tok informácií medzi oddeleniami prispeje k zníženiu chýb spôsobených nedostatočnou komunikáciou a zvýši koordináciu medzi jednotlivými útvarmi.

\subsection*{Strategický rozvoj a digitalizácia podniku}

Zavedenie \gls{dms} predstavuje krok smerom k systematickej digitalizácii podniku a posilneniu jeho informačnej infraštruktúry. 
Vytvorením jednotnej informačnej základne sa vytvára priestor pre budúcu integráciu s ďalšími podnikovými systémami, napríklad s ekonomickým softvérom alebo \gls{erp} riešením.

Dlhodobým prínosom môže byť vytvorenie digitálnej znalostnej databázy projektov, ktorá umožní efektívnejšie využívanie skúseností z minulých realizácií. 
Takýto prístup podporuje inovačný potenciál podniku a zvyšuje jeho konkurencieschopnosť na trhu.

\subsection*{Celkové zhodnotenie prínosov}

Implementácia systému správy dokumentov predstavuje investíciu do organizačnej stability, informačnej bezpečnosti a systematického riadenia znalostí v podniku. 
Hoci zavedenie systému vyžaduje počiatočné náklady a adaptáciu zamestnancov, očakávané prínosy v oblasti efektivity, transparentnosti a riadenia rizík prevyšujú tieto náklady v strednodobom horizonte.

\clearpage