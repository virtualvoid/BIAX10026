\section{Návrh riešenia}

Na základe identifikovaných požiadaviek a analýzy súčasného stavu bol navrhnutý systém správy dokumentov 
ako samostatný modul podnikového informačného systému. 
Návrh reflektuje prioritné požiadavky kategórie Must (F1–F3, N1–N2, B1–B3) 
a zároveň vytvára architektonický základ pre budúce rozšírenie a integráciu (F4–F5, N3, I1–I2).

\subsection*{Architektúra systému}

Navrhované riešenie \gls{dms} je koncipované ako viacvrstvová architektúra, 
ktorá oddeľuje prezentačnú, aplikačnú a perzistenčnú vrstvu. 
Takéto členenie umožňuje jasnú separáciu zodpovedností, 
zvyšuje udržiavateľnosť systému a podporuje jeho postupné rozširovanie.

Logická architektúra je znázornená na Obrázku~\ref{fig:architecture}.

\begin{figure}[ht]
    \centering
    \includegraphics[width=0.95\textwidth]{images/architecture.png}
    \caption{Logická architektúra navrhovaného systému správy dokumentov}
    \label{fig:architecture}
\end{figure}

\subsubsection*{Prezentačná vrstva}

Prezentačná vrstva poskytuje používateľské rozhranie dostupné prostredníctvom webového prehliadača. 
Komunikácia je zabezpečená protokolom \gls{https}.

Táto vrstva umožňuje:

\begin{itemize}
    \item evidenciu a vyhľadávanie dokumentov (F1, F4),
    \item nahrávanie nových verzií dokumentov (F2),
    \item realizáciu workflow krokov (F3),
    \item správu metadát a väzieb na projekt.
\end{itemize}

Prezentačná vrstva neobsahuje podnikové pravidlá, 
ale funguje ako rozhranie medzi používateľom a aplikačnou logikou. 
Tým sa zabezpečuje lepšia testovateľnosť a možnosť budúcej výmeny technológie používateľského rozhrania.

\subsubsection*{Aplikačná vrstva}

Aplikačná vrstva predstavuje jadro systému. 
Implementuje podnikové pravidlá, riadi životný cyklus dokumentov a zabezpečuje:

\begin{itemize}
    \item centrálne riadenú evidenciu dokumentov (F1),
    \item riadené verzovanie vrátane histórie zmien (F2),
    \item workflow schvaľovanie dokumentov (F3),
    \item kontrolu prístupových práv na princípe \gls{rbac} (B1),
    \item generovanie auditnej stopy (B2).
\end{itemize}

Táto vrstva zabezpečuje validáciu vstupov, konzistenciu údajov 
a kontrolu prechodov medzi stavmi dokumentu (napr. návrh – schválený – archivovaný). 
Zároveň implementuje bezpečnostné mechanizmy a kontrolu oprávnení.

Aplikačná vrstva je navrhnutá tak, aby umožňovala rozšírenie 
o integračné rozhrania (I1, I2) bez zásahu do prezentačnej časti.

\subsubsection*{Perzistenčná vrstva}

Perzistenčná vrstva zabezpečuje uchovávanie štruktúrovaných údajov 
a fyzických súborov dokumentov.

Štruktúrované dáta (metadáta, používateľské účty, role, workflow stavy, auditné záznamy) 
sú ukladané v relačnej databáze. 

Samotné súbory dokumentov sú ukladané do samostatného úložiska 
(file storage alebo objektové úložisko), čím sa dosahuje:

\begin{itemize}
    \item lepšia škálovateľnosť pri práci s veľkými technickými súbormi (N3),
    \item jednoduchšie zálohovanie a obnova dát (N2),
    \item oddelenie logickej a fyzickej vrstvy dát.
\end{itemize}

\subsection*{Prepojenie architektúry s identifikovanými požiadavkami}

Navrhnutá architektúra priamo reflektuje identifikované požiadavky:

\begin{itemize}
    \item F1 (centrálna evidencia) je implementovaná v aplikačnej vrstve 
    prostredníctvom jednotného registra dokumentov.
    
    \item F2 (verzovanie) je zabezpečené samostatnou entitou verzie dokumentu 
    s historickou evidenciou zmien.
    
    \item F3 (workflow) je realizované prostredníctvom stavového modelu 
    a definovaných prechodov medzi stavmi.
    
    \item B1 (RBAC) je implementované na úrovni aplikačnej logiky 
    a kontrolované pri každej operácii.
    
    \item B2 (auditná stopa) je generovaná ako samostatná perzistentná entita.
    
    \item N2 (zálohovanie a obnova) je podporené oddelením databázovej 
    a súborovej vrstvy.
    
    \item I1 a I2 (integrácia) sú umožnené návrhom aplikačnej vrstvy 
    s potenciálom poskytovať API rozhrania.
\end{itemize}

\subsection*{Návrh dátového modelu}

Kľúčové entity systému zahŕňajú:

\begin{itemize}
    \item \textbf{Projekt} – základná organizačná jednotka, ku ktorej sú dokumenty viazané.
    \item \textbf{Dokument} – logická reprezentácia dokumentu (typ, názov, väzba na projekt).
    \item \textbf{Verzia dokumentu} – konkrétna verzia dokumentu s časovou pečiatkou a autorom.
    \item \textbf{Používateľ} – osoba oprávnená pristupovať do systému.
    \item \textbf{Rola} – definícia oprávnení v rámci modelu \gls{rbac}.
    \item \textbf{Workflow} – definícia stavov a prechodov dokumentu.
    \item \textbf{Auditný záznam} – evidencia vykonaných operácií.
\end{itemize}

Vzťahy medzi entitami umožňujú:

\begin{itemize}
    \item väzbu dokumentov na projekty,
    \item viacnásobné verzie jedného dokumentu,
    \item riadenie oprávnení podľa rolí,
    \item spätnú dohľadateľnosť zmien.
\end{itemize}

Takto navrhnutý model podporuje konzistentnosť údajov 
a umožňuje budúce rozšírenie (napr. prepojenie dokumentu na ekonomické doklady).

\subsection*{Bezpečnostné aspekty návrhu}

Bezpečnostná architektúra zahŕňa:

\begin{itemize}
    \item šifrovanú komunikáciu cez \gls{https},
    \item autentifikáciu používateľov,
    \item autorizáciu na princípe \gls{rbac},
    \item auditnú stopu všetkých významných operácií.
\end{itemize}

Prístupové práva sú kontrolované na úrovni aplikačnej logiky, 
čím sa minimalizuje riziko neoprávneného zásahu (B3).

\subsection*{Škálovateľnosť a budúce rozšírenie}

Navrhnutá architektúra umožňuje horizontálne aj vertikálne škálovanie systému.

Oddelenie vrstiev umožňuje:

\begin{itemize}
    \item rozšírenie aplikačnej vrstvy o ďalšie moduly (napr. registratúrny modul),
    \item implementáciu REST API pre integráciu s ERP (I1),
    \item rozšírenie o pokročilé vyhľadávanie alebo indexáciu obsahu,
    \item integráciu s externými úložiskami alebo cloudovou infraštruktúrou.
\end{itemize}

V prípade rastu objemu dokumentácie je možné:

\begin{itemize}
    \item oddeliť databázový server,
    \item distribuovať súborové úložisko,
    \item zaviesť cache mechanizmy.
\end{itemize}

Architektúra tak vytvára stabilný základ pre postupnú digitálnu transformáciu podniku.

\subsection*{Zhrnutie návrhu riešenia}

Navrhnuté riešenie \gls{dms} predstavuje systematickú odpoveď 
na identifikované nedostatky súčasného stavu. 
Architektúra rešpektuje princípy modulárnosti, bezpečnosti a škálovateľnosti.

Systém pokrýva prioritné funkčné a bezpečnostné požiadavky, 
pričom zároveň vytvára predpoklady pre budúcu integráciu do širšieho \gls{pis}.

\clearpage
